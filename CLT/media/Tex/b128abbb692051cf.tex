\documentclass[preview]{standalone}

\usepackage[english]{babel}
\usepackage{amsmath}
\usepackage{amssymb}

\begin{document}

\begin{center}
$\overline x_n = \displaystyle \frac{1}{10}(0.54 + 0.99 + 0.32 + 0.27 + 0.14 + 0.44 + 0.90 + 0.32 + 0.85 + 0.75) = 0.55$
\end{center}

\end{document}
